% Options for packages loaded elsewhere
\PassOptionsToPackage{unicode}{hyperref}
\PassOptionsToPackage{hyphens}{url}
\documentclass[
]{article}
\usepackage{xcolor}
\usepackage{amsmath,amssymb}
\setcounter{secnumdepth}{-\maxdimen} % remove section numbering
\usepackage{iftex}
\ifPDFTeX
  \usepackage[T1]{fontenc}
  \usepackage[utf8]{inputenc}
  \usepackage{textcomp} % provide euro and other symbols
\else % if luatex or xetex
  \usepackage{unicode-math} % this also loads fontspec
  \defaultfontfeatures{Scale=MatchLowercase}
  \defaultfontfeatures[\rmfamily]{Ligatures=TeX,Scale=1}
\fi
\usepackage{lmodern}
\ifPDFTeX\else
  % xetex/luatex font selection
\fi
% Use upquote if available, for straight quotes in verbatim environments
\IfFileExists{upquote.sty}{\usepackage{upquote}}{}
\IfFileExists{microtype.sty}{% use microtype if available
  \usepackage[]{microtype}
  \UseMicrotypeSet[protrusion]{basicmath} % disable protrusion for tt fonts
}{}
\makeatletter
\@ifundefined{KOMAClassName}{% if non-KOMA class
  \IfFileExists{parskip.sty}{%
    \usepackage{parskip}
  }{% else
    \setlength{\parindent}{0pt}
    \setlength{\parskip}{6pt plus 2pt minus 1pt}}
}{% if KOMA class
  \KOMAoptions{parskip=half}}
\makeatother
\setlength{\emergencystretch}{3em} % prevent overfull lines
\providecommand{\tightlist}{%
  \setlength{\itemsep}{0pt}\setlength{\parskip}{0pt}}
\usepackage{bookmark}
\IfFileExists{xurl.sty}{\usepackage{xurl}}{} % add URL line breaks if available
\urlstyle{same}
\hypersetup{
  pdftitle={Reimann Integration},
  pdfauthor={Vaishnav Sudarshan},
  hidelinks,
  pdfcreator={LaTeX via pandoc}}

\title{Reimann Integration}
\author{Vaishnav Sudarshan}
\date{}

\begin{document}
\maketitle

\section{Integration}\label{integration}

Suppose we want to find the area bounded by under a function and the
\(x\) axis, from \(x = a\) to \(x = b\).

The shape between the function and the \(x\) axis isn't a shape that you
can find the area of using geometry. So, we have to find a way to split
the area into infinite small pieces, similar to how the formula for the
area of a circle is derived.

An easy way to split this area is to divide the \(x\) axis from \(a\) to
\(b\) into some number of widths. Each width will be the width of a
rectangle, and the height of the rectangle is just the value of the
function evaluated at the start of that width. Adding up the areas of
all these rectangles is a good approximation for the true area. We can
write this mathematically as:

If these widths get smaller and smaller, which also means there are more
rectangles, this approximation becomes better and better. This means
that the exact area under the curve is the sum of the areas of all the
rectangles as the widths approach 0. Since each width approaches 0, and
each is also a change in x, we can write it as dx. To write this
mathematically, \[
A = \int\limits_{a}^{b} f(x) dx
\] The weird symbol that looks like an \(s\) is the integral symbol.
Just like how d is like delta but infinitely small, the integral symbol
is like sigma except the values being added up are infinitely small.

\end{document}
